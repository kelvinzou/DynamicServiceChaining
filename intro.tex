 \begin{comment}
\section{Motivation}

% What do you mean by Internet can move? It sounds weird.
Today's Internet can move at any  point. The soaring demands of client
mobility  over   the  last two decades,    the   growing elasticity of
virtualized server, and the  recent trends of elastic network function
virtualiztion,   impose  new challenges   for  network  infrastructure
design.

% Change the changeable, accept the  unchangeable, and remove yourself
% from the unacceptable.--Denis Waitley

Many mobility  solutions rely on  an  artificial choke-point,  such as
name server or home agent, which  are installed as middleboxes in many
places. Recent trends of network  function virtualization aim to offer
a more scalable and flexible middlebox functions, and network function
migration can happen at any  time. We raise  a question, can we have a
unified solution that can support every point (e.g., endhosts, network
functions, server VMs) moving in the network?

Although previous approaches consider some of  the problem we address,
none addresses all this issues in a unifying solution.

\end{comment}
 
%%%%%%%%%%%%%%%%%%%%%%%%%%
%%%%%INTRODUCTION
%%%%%%%%%%%%%%%%%%%%%%%%%%

\section{Introduction}

\textit{Middleboxes    are   ubiquitous     and   elastic}:    Network
administrators   use middleboxes to     apply services on the  traffic
exchanged  between pairs of  communicating  endpoints. Today, they are
routinely  used   to improve  security   (firewall,  packet scrubber),
protect  user privacy (encryption,  anonymization),  share a set of IP
addresses (network address  translator), spread traffic over  multiple
backend  servers  (load   balancing),   reduce  bandwidth  consumption
(compression, video transcoding, caching), monitor traffic and perform
application-specific    tasks.   Recent trends   of network   function
virtualization (NFV) implement   middleboxes  as virtual  machines (VMs)  or
user-space  applications separate     from   the  physical      host.
Virtualization enables running network functions (NFs) at many different locations
in the  network or even in public  clouds, and the  NF instance
can spin up (or down) or even migrate as load demands.

%, Internet of things such as connected  cars over the past few years 


\textit{Endpoints move}: The growing cellular network and WiFi hotspot
coverage has  pushed  the term   ``mobile''  to a   new era.  Mobility
support has  become  a   key   consideration   in terms  of    network
infrastructure  design. More and  more network applications and services  are run in
virtualized environments; the server VMs  can run anywhere and the VM
migration    can   happen at  any  moment    due   to load  balancing,
infrastructure maintenance, etc.

% \amy{As you probably know,  this paragraph is improperly placed. But
%   we can move it around later.}
To  support a new Internet  era where client  mobility, and server and
middlebox migration can take place anywhere and  anytime, we propose a
\system (definition goes here).
% that sits below an unmodified transport layer  and requires no router or  switchsupport. The core of  the architecture  is  a new geomorphic view of today's  network.  

The  existence  of middleboxes  already  breaks the
end-to-end principle, so we capitalize  on this violation to create  a
new  abstraction for a connection.  We  identify  each connection by a
supersession and  a list  of   subsessions, where  a   supersession
represents  the original  end-to-end communication. NF  traffic
steering simply stitches  the subsessions together, and  mobility and
migration apply the  same  building block  operation:  replacing an old
subsession with a new  one. Hence, the subsession abstraction exists
to   isolate  middlebox operations in   a  supersession to individual
subsessions.  



% \amy{This concluding  sentence is a bit  awkward-- I  don't know the
%   convention for this.}.

Although previous works consider some of the problems we address, none
provides  a  comprehensive  solution  for   endpoint  mobility, server
migration    and  middlebox  service     chain  and  migrations.   TCP
Migrate~\cite{TCPMobile}, Mobile IP~\cite{mip},   Serval~\cite{serval}
and ROAM~\cite{I3Mobile}  support endpoint mobility,  but none of them
explicitly deals  with the existence  of middleboxes. DoA~\cite{DOA} proposes a delegation architecture
for   middlebox  service  chain,   however  it  neither  supports flow
migration     across     NF       nor    server    migration.
SIMPLE~\cite{SIMPLE} and OpenNF~\cite{OpenNF} take advantage of modern
switches that offer fine-grained control  over routing (e.g., OpenFlow
enabled  switch),  to steer traffic selectively  through  one  or more
middleboxes.  However, the  routing solutions fail to support endpoint
mobility and  suffer from scalability  and flexibility of fine-grained
policy control due to the limited  size of ternary content-addressable
memory  (TCAM) on switches.  More  fundamentally, we argue middleboxes
should be  addressed explicitly  instead  of treated  as  second-class
citizens.

% \amy{It is unclear thus far how  you are supporting client mobility.
%   You simply  mention it in  the previous paragraph. ech, maybe this
%   discomfort will go away when you put  more in the intro. Basically
%   the previous paragraphs are awkward  because you go in depth  into
%   what your  architecture  is, but  only on  the  middlebox protocol
%   part. If  you're going to  describe things, might as well describe
%   the whole architecture and explain how  you are achieving all your
%   goals}

%In  \system middleboxes are  addressed explicitly  as the endpoints of subsessions  and  thus  can   be  an  integral  part  of  the mobility ecosystem. 

\system achieves:

\begin{itemize}
\item \textbf{Unified solution:} It  supports client mobility,  server
  migration,   middlebox  service  chain  and   flow  migration during
  middlebox migration.
\item  \textbf{Incrementally  deployable:} The system  can be deployed
  between client  and server, client   and  middleboxes, or even  just
  between middleboxes.
\item \textbf{Same socket abstraction:}  It does not change  transport
  layer,  and thus does   not require any   change to the higher-level
  application if the application is using socket abstraction.
\item  \textbf{No routing change:} Since  we  address the sub-sessions
  explicitly by the network layer, simple solutions like shortest path
  routing still work.
\item  \textbf{High performance:} An  in-kernel prototype shows that a
  shim translate layer can sustain up to 14.2 Gbps per CPU core.

% \amy{This shim translate layer just kind  of popped up out of the
%   blue. Do you think it is necessary to  mention it briefly prior
%   to this?}

\end{itemize}



%Most mobility proposals decouple IP and ID binding, by introducing an indirection point (e.g., Mobile IP~\cite{mip}), redirecting through DNS (e.g., TCP Migrate~\cite{TCPMobile}), or resolving persistent ID via a distributed hash table (DHT) (e.g.,ROAM~\cite{I3Mobile}). However, the solutions either lack the support for Middleboxes (e.g., ROAM) or requires an implicit fixed middlebox (e.g., home agent in Mobile IP). 




% However, getting middleboxes on (and off) a network path is unnecessarily complicated. Traditionally, middleboxes are dedicated appliances deployed at major juncture points, such as the gateway that connects an enterprise to the Internet. But this approach makes it difficult to apply middleboxes to internal traffic (staying within an enterprise or data center) or to use middleboxes selectively (e.g., allowing non-video traffic to bypass the transcoder).

% Two recent trends are making middlebox placement much more flexible. First, middleboxes are increasingly virtualized---implemented as virtual machines (VMs) that are separable from the physical host. This means that middleboxes can run at many different, strategic locations in the network, and the middlebox instances can spin up (or down) or even migrate as load demands. Second, modern switches offer fine-grained control over routing (e.g., using technologies like OpenFlow) to steer traffic selectively through one or more middleboxes. For example, web requests may traverse a firewall followed by a load balancer, while video traffic may traverse a firewall followed by transcoder. Together, these two trends turn the network into a fungible pool of resource for running services and steering traffic. 

\begin{comment}
\subsection{Middlebox-Aware Session Protocol}
We see  a growing number  of research papers  and industry activity on
controllers  that  perform flexible  ``traffic steering" (or ``service
chaining")  to optimize   performance  and/or  enforce policy.  Almost
exclusively, these controllers use \textit{routing} as their mechanism
of  control.  In the  worst  case, the  controller installs customized
``microflow" rules on  demand to  direct all the  packets of  each TCP
session through the chosen sequence of middleboxes. Alternatively, the
controller could proactively install  coarse-grained rules that handle
related TCP sessions in  the same way, at  the expense of less control
and greater complexity.    However, we argue  that  it  is  clumsy and
inefficient to use routing to steer different sets of sessions through
different     chains    of  middleboxes;     as    it   suffers   from
(i)\textit{scalability} and (ii)\textit{mobility}.

We argue  that middlebox-aware session protocols   (MBPs) are a better
way  to insert   and remove middleboxes  on   a path than  routing is.
Briefly, session protocols  do   not have scalability   or flexibility
problems  because the information  required  to implement each session
correctly is associated with the session itself.  For the same reason,
it  is easy to ensure session  affinity and dynamically add, remove or
swap middleboxes.  We  store  the  middlebox  policy at  a   logically
centralized policy server, which  functions like a  DNS server  or SDN
controller.   To achieve   this,  we  have to   address the  following
challenges:
%A simple comparison between MBPs and other existing systems is in Table~\ref{compare}. 

\begin{itemize}

\item \textbf{Scalable policy routing:} Due to  a limited size of flow
  table,  it is hard  to  put  in tens of    thousands to millions  of
  ``microflow" rules in to router or switches' hardware routing table.
  How to decompose  the middlebox  policies and offload   fine-grained
  policies to  end hosts and middleboxes  in an efficient way is still
  an open problem.

\item \textbf{Session affinity during flow migration:} Flows should be
  flexibly routed   between  different  middleboxes,   however,   this
  requires either maintaining a   per-session state in the  switch, or
  having complex ``time  out" rules to keep  states in the controller.
  It is important to have  a middlebox protocol which guarantees  flow
  affinity, however concurrency and  race condition make this  problem
  more challenging.

\item \textbf{Endpoint mobility   and multihoming:} Today's  end hosts
  change locations frequently due to device mobility and VM migration.
  In addition, newer  protocols   allow end devices to   use  multiple
  interfaces. We also need to address how  to seamlessly integrate end
  hosts mobility protocols and this MBP.

\end{itemize}
\end{comment}



The paper is organized as follows:
Section 2 gives motivating examples of dynamic service chaining and endhost mobility and how current solutions fail to meet the requirement. 
The architecture of \system is described in Section 3, and the protocol of supersession -- subsession establishment and migration is described in Section 4. 
Section 5 describes crucial data plane properties provided by \system. 
Implementation is described in Section 7 and evaluation in Section 7. 
Section 8 presents a discussion about deployment, security and fault tolerance. 
We cover related work in Section 9 and conclude in Section 10.   
