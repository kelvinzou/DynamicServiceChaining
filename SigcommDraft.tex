
\documentclass{sig-alternate-10pt}
\usepackage{url}
%\usepackage[protrusion=true,expansion=true]{microtype}  
\usepackage{graphicx} 
\usepackage{wrapfig} 
\usepackage{amsmath}
\usepackage{comment}
\usepackage{cleveref}
\usepackage{float}
\usepackage{color}
\usepackage[ruled,noend]{algorithm2e}
\usepackage{footnote}
\usepackage[font=small,labelfont=bf]{caption}

\usepackage{pifont}% http://ctan.org/pkg/pifont
\newcommand{\cmark}{\ding{51}}%
\newcommand{\xmark}{\ding{55}}%
\makesavenoteenv{tabular}
\makesavenoteenv{table}
\newcommand{\system}{M2M}

\newcommand{\kelvin}[1]{\textcolor{blue}{(KELVIN: #1)}}
\newcommand{\amy}[1]{\textcolor{red}{(AT: #1)}}

\makeatletter


%\title{\LARGE\textbf{Rethinking How to Deploy In-Network Services}}
\title{\LARGE\textbf{Dynamic Service Chaining with \system}}


\author{\textsc{Xuan Kelvin Zou, Amy Tai, Ronaldo Ferreira, Jennifer Rexford, Pamela Zave} }

\begin{document}

\maketitle  
\begin{abstract}


  Middleboxes  are    at once an  abomination    (a  violation  of the
  end-to-end argument)  and a  necessity (for  deploying  higher-level
  services between  endpoints).  Conventional  techniques  for getting
  middleboxes ``on path'', so called network service chaining, rely on
  manipulating how  the  network  switches  route traffic.    However,
  tweaking  the  routing  configuration   is clumsy  and  inefficient,
  especially when middleboxes  need to serve   mobile endpoints or  to
  support dynamic flow migration.  Instead, we propose an architecture
  for  Mobility and Migration  Management (M2M).  M2M presents a clean
  separation of data, control,  and management planes for scalable and
  flexible policy  management with  low   overhead.  We  introduce   a
  Middlebox-to-Middlebox   (M-to-M)  protocol that   leverages M2M for
  dynamic service  chaining.   The     M-to-M protocol   deals    with
  middleboxes as  an explicit part  of the  end-to-end path, and adds,
  removes   or replaces middleboxes  dynamically  as  needed.  We also
  develop update  algorithms to  address  packet reordering  and  race
  conditions during a  flow migration. Our  high-performance prototype
  implementation  can sustain 40  Gbps    on a commodity server,   and
  establish  a session and manage  flow migrations within the order of
  milliseconds.

% With middleboxes and mobility all increasingly the norm, we argue that it is high time to push for a clean, unified solution that solves all of these problems in the right place$(G!7(B the session protocol$(G!7(Brather than the routing system.

 
%Instead, we propose a middlebox-aware session protocol that makes middleboxes an explicit part of the end-to-end path, and adds or removes middleboxes dynamically as needed. We observe that session-location mobility protocols$(G!7(Bwhich maintain a session across changes in endpoint locations$(G!7(Bnaturally extend to support inserting, removing, and replacing mid- dleboxes. Our protocol enables a mix of signaling and non- signaling middleboxes, and can easily run in a shim layer below TCP for lower overhead and easier deployment. With middleboxes, mobility, and multihoming all increasingly the norm, we argue that it is high time to push for a clean, unified solution that solves all of these problems in the right place$(G!7(B the session protocol$(G!7(Brather than the routing system.


% In addition to delivering data efficiently, today's computer networks often perform services on the traffic in flight to provide new features, enhance security, privacy, or performance. Network administrators frequently install so-called ``middleboxes'' such as firewalls, network address translators, server load balancers, Web caches, and devices that compress or encrypt the traffic. Recent trends of Network Functions Virtualization (NFV) and Software Defined Networks (SDN) turn the network into a fungible pool of resource for running services and steering traffic. Based on a NFV setting, we propose an optimization solution. More specifically, we introduce and study a new class of multi-commodify flow problems where in addition to demands on flows and capacity constraints on edges, there is an additional requirement that flows be processed by middleboxes in the network. 

% Current techniques for getting middleboxes ``on path" rely on manipulating how the network switches route traffic. However due to flow session affinity and routing policy conflicts, tweaking the routing configuration to comply with a network is clumsy and inefficient. To achieve the routing solution we obtain from the optimization scheme, we propose a middlebox-aware session protocol (MBP) that makes middleboxes an explicit part of the end-to-end path, and adds, removes or replaces middleboxes dynamically as needed. This approach is easiest to achieve in a cloud environment, where---not coincidentally--the need for scalable NFV is greatest. 
\end{abstract}


\input{intro.tex}
%%%%%%%%%%%%%%%%%%%%%%%%%%
%%%%%Related work
%%%%%%%%%%%%%%%%%%%%%%%%%%
\begin{comment}
\begin{table*}[ht]\label{compare} 

\small 

\begin{tabular} {|l |c | c| c| c| c| c| c| }

\hline

                            			     &Scalable Fine-         & Supersession     & No Routing    	 &   Interdomain    &    Middlebox        &  Endpoint  &  No Application\\
						     &Grained Control        & Liveness     	& Modification  	 &   Middlebox       &Migration       &  Mobility  & change \\\hline
SIMPLE,FlowTags~\cite{SIMPLE,FLOWTAGS}        	   & $\oslash$             & $\oslash$   	& $\otimes$     	 &   $\oslash$      &    $\oslash$       &  $\otimes$   \\ \hline
OpenNF,Split/Merge~\cite{OpenNF,splitmerge}         & $\otimes$             & $\odot$     	& $\otimes$     	 &   $\oslash$      &    $\odot$         &  $\otimes$  \\ \hline
CoMB~\cite{CoMB}             & $\odot$               & $\odot$     	& $\odot$       	 &   $\otimes$        &    $\otimes$       &  $\otimes$ \\ \hline 
DoA~\cite{DOA}               & $\otimes$             & $\odot$    	 & $\odot$     	 	  &   $\odot$        &    $\otimes$       &  $\otimes$  \\ \hline
APLOMB~\cite{Aplomb}         & $\otimes$             & $\otimes$  	 & $\otimes$  	   	 &   $\odot$        &    $\odot$         &  $\otimes$ \\ \hline
Software Routing~\cite{OVS, click} & $\odot$         & $\otimes$  	 & $\odot$      	  &   $\odot$        &    $\oslash$       &  $\otimes$  \\ \hline



\end{tabular}
\caption{\small Comparison between different middlebox traffic steering solutions and mobility protocols ($\odot$ indicates that a scheme fully supports the criterion; $\otimes$ indicates a scheme does not, $\oslash$ indicates partially support or can be extended to) }
\end{table*}
\end{comment}




\section{Motivating Examples}

In this section, we first present a few scenarios where a system may require network function (NF) insertion, removal, or migration, and host mobility. We also discuss how the current solutions fail to address our requirements.


\subsection{Dynamic NF Policy}
Enterprise networks deploy various network functions for better performance and security. Being able to modify dynamically NF types/instances in a service chain improves the efficiency and flexibility of the network system. NF insertion is necessary if a flow is marked as suspicious by a coarse-grained intrusion detection and prevention system (IDPSs~\cite{IPS}) and requires a fine-grained deep packet inspection (DPI). NF removal is preferred if a connection goes through a cache proxy, but the proxy has a cache-miss and the content is uncacheable; the cache proxy may remove itself from the chain. NF load balancing is necessary if the network operators want to distribute a flow evenly among multiple instances of the same NF. Moreover, NF instance migration can happen in a network function virtualization (NFV) setting, and the flows that use the NF instance also need to migrate along with the instance. 

% \subsection{Case Study}
\begin{figure}[ht]
\centering
\includegraphics[width=\linewidth]{figures/routingsucks.pdf} 
\caption{\small The distribution of flow size for different prefix can be
measured using a weight\protect\cite{Niagara}. We assume two different weight
distributions for IP prefix: Gaussian, Bimodal  Gaussian with the parameters from\protect\cite{Niagara}. 
We show that in the  case of distributing flow over two  NF instances, a
routing solution, while doubling rules, fails to balance the load across middleboxes in the scenario
above. (Prefix split uses the most significant bit, and optimal split uses arbitrary one bit wildcard matching).} \label{distribution}
\end{figure}
% 


Efficient, dynamic NF policies cannot be implemented on conventional switches due to coarse-grained routing, e.g., the switch may simply tunnel all the traffic to IDS for further inspection and remove the IDS once it finds the proverbial ``needle in the haystack''. Some recent techniques leverage fine-grained routing switches (e.g., OpenFlow~\cite{openflow} based) for dynamically changing NF policies. However, these solutions are neither scalable due to TCAM rule size, nor flexible due to a complex dependency between routing rules. The SDN controller plays an excessive role in data plane, while still fails to fulfill certain network order-preserving properties~\cite{splitmerge, OpenNF}. Figure~\ref{distribution} gives an example of the imbalance caused by routing solution when trying to distribute flows across two instances of the same NF type. 

\subsection{Mobility and NF Policy} 
A cellular network can be divided into user equipment (UE), local access network (LAN) and core network.  LANs communicate to UEs through its base station and communicate to the Internet through the core network. Cellular networks rely on a wide range of NFs (e.g., firewall~\cite{IPTABLES}, load balancer~\cite{balance}, cache proxy~\cite{squid}) to improve performance and enhance security. As mobility and network function virtualization become ubiquitous, we may ask for (i) seamless mobility and (ii) dynamic service chaining.

Many mobility solutions have been proposed over the past years~\cite{mip, TCPMobile, I3Mobile, serval}, yet none of them consider the existence of network functions. Worse, NFs can even be a hindrance for these protocols since they may interfere with the protocol control logic (e.g., TCP split~\cite{TCPProxy}). Routing-based solutions have been proposed for service chaining in cellular networks(e.g., SoftCell~\cite{softcell}). However, in order to support host mobility, it places NFs only in the core network and does not support dynamic service chaining. 



%\pagebreak
\section{Architecture}\label{sec:arch}

We now  give  an overview of   \system architecture.  We  discuss  the
design goals that    motivate  a new architecture, which     separates
functionality  into three different  planes,  architecture components,
and the mechanisms to achieve the design goals.
\subsection{Design Goals}

{\bf  Scalable   policy management.}    Correctness conditions dictate that traffic must be correctly forwarded through a middlebox chain according to network policies.  Relying on routing
for traffic steering  is inefficient, because distributed
routing protocols are slow to converge during topology reconfiguration
and   provide only  coarse-grained   control  over  flows.   SDN-based
solutions  avoid this problem  by installing per-flow rules on network
switches, but these do not scale well since the SDN controller must be
involved  in per-flow decisions, and switches  have  limited memory to
store flow rules; in the worst case, each switch must install one rule
per TCP flow.

In \system,    we avoid both of    these scalability  impediments. The
protocol is designed such that the destination  address of each packet
identifies the next middlebox  or endpoint in  the path and the source
address identifies the sending  middlebox or endpoint. Doing so avoids
the need of routing tweaks  when network topology changes or endpoints
move.    Policy  management in \system  is   controlled by a logically
centralized controller,  but  packets  on the   data  plane are  never
redirected  to the policy server.   This is a  key difference from SDN
solutions  that queue packets  on the controller during flow migration
between middleboxes~\cite{OpenNF}.   Moreover, the  policy  server can
offload per-flow  decisions   to the middleboxes, for   instance allowing a
middlebox  to remove  itself  from  a session
path. Off-loading further relaxes any dependence on the policy server.

{\bf   Low  performance overhead.}   Recall    that we  identify  each
connection  with a   supersession and  its   associated subsessions. A
simple approach for decomposing a supersession into subsessions can be
implemented by establishing  tunnels between middeboxes.   However, by
adding a  new  header  to   each  packet,  this  approach   introduces  overhead   in the form of MTU  increase    and consequently  packet
fragmentation. Packet fragmentation  can  be solved by  increasing the
MTU of switches  and routers inside  a single  network administration,
but this  solution is not  viable when packets  have  to cross network
domains,  which   has become   common   in virtualized  network
functions that are outsourced to  public clouds.  Moreover,  solutions
that rely on tunnels generally add extra  overheads such as encryption
and compression~\cite{Aplomb}, features  that  can be   redundant or
unnecessary  to   all   flows.  \system   relies   on network  address
translation for explicitly  addressing subsession endpoints, hence the
only overhead is incurred by  port remapping.  Also, supersession  IDs
in \system are not carried on each  packet, as in~\cite{DOA}. They are
stored   on the   middlebox   agents  during  supersession   setup  and reconfiguration.

{\bf  Endpoint  mobility and  NF   migration.}  
%\amy{How  about  renaming    this   sub point    as   ``Separation  of 
%  Functionality'', or   something  that references the   separation of
%  decision-making into 3 planes, or  even something about  simplicity.
%  Separating planes so that  you  can cleanly implement  host mobility
%  and NF migration. In any case, the current title is meaningless.}
The growing trend in network function virtualization can cause dynamic
migration of flows between  middleboxes. Along with frequent  end host
location changes  due to device mobility  or  VM migration,  these new
possibilities  complicate     traffic-steering  solutions  relying  on
administrative control over network routing.  In \system architecture, dynamic policy changes are centrally abstracted by
a  separate policy  server, which decides  (i)   the sequence of  middleboxes
traffic should traverse and  (ii) informs the hosts --- endpoints
or  middleboxes --- of these decisions by pushing  policies ahead of time
to avoid increasing connection set-up  delay.  After fulfilling these two responsibilities, the policy server steps back  to allow  hosts to establish sessions and handle
migrations as needed.

% Incremental deployment
% This will be moved to a different subsection
% 1. Does not rely on a special  naming infrastructure. It can use any
% scheme,  as  long as   the   endpoints can   be  identified  in  the
% subsessions. A different naming scheme, however, requires applications
% and NFs changes.
% 2.  Does not require all the   components to be  integrated into the
% architecture for initial deployment. 
% 3. Does not require application or NF changes.

\subsection{Components}

An overview of  the \system architecture  with its main components and
the    interactions     between   the    modules    is    depicted  in
Figure~\ref{architecture}.   The  architecture     presents    a clear
separation of  management,  control, and  data plane functionalities.
The management  plane   is composed   of  a   policy server that    is
responsible for coordinating the agents on  the control plane based on
high-level network policies provided by the network administrator. The
control  plane implements   the protocol  in  \S\ref{sec:protocol} for
session initiation, flow migration, and network function insertion and
removal.   The data plane is  responsible for translating supersession
IDs  into local IP addresses    and ports, delivering  packets to  the
network functions,  and forwarding   packets between  middleboxes.  To
simplify  the presentation, we first  assume a homogeneous scenario in
which endpoints  and all middleboxes run \system  agents, and defer to
\S\ref{sec:discussion} a discussion of a more realistic scenario where
\system  can be deployed incrementally.  Also,  we assume that network
functions  do  not   change  the  packet  5-tuple, i.e. ---  source and
destination IP addresses  and  port numbers,  and protocol type;  the
case    where     packets  are   changed    will    be   discussed  in
\S\ref{sec:NFsupport}.

\begin{figure}[hb]
\includegraphics[width=\linewidth]{figures/archi.pdf}
\caption{\small\system architecture}\label{architecture}
\end{figure}

% Load   balancing:    Due to the    complex   packet  processing that
% middleboxes run  (e.g.,  deep packet  inspection),  a key  factor in
% middlebox deployments   is to balance the   processing load to avoid
% overload.

% To avoid inference algorithms, which  do not give 100 precision,  we
% may require  NF modifications so they  can interact with the \system
% agents.

% Typical middlebox policies require a packet (or session) to traverse
% a sequence of middlebox (This is  an instance of the broader concept
% of "service chaining").  In our example,  the administrator wants to
% route all HTTP traffic through the policy chain FW-IDS-PROXY and the
% remaining  traffic   through  the   chain FW-IDS.  Note   that  many
% middleboxes  are stateful and need to  process  both directions of a
% session for correctness.

% Middlebox resource management:  Studies show that middlebox overload
% is a  common cause of failures.  

% "The  policy server can take global   decisions to avoid overload or
% give multiple options to the \system agents  so they can balance the
% load across their neighbors.

% SIMPLE 

{\bf Policy  server:}   The  \system  policy  server is   a  logically
centralized server that receives  high-level policy specification from
the  network  administrator and reliably   delivers subpolicies to the
agents, i.e. --- it makes sure that an agent receives a policy consistent
with its location.  For instance,  a client agent should receive  only
policies related to the network it is connected.

The  policy server makes  global decisions based on: (i) configuration
changes  or (ii)   network state changes.    Configuration changes are
triggered  by the network administrator, and  network state changes are
triggered  by monitoring   information, which the  policy  server  receives  from the \system   agents. Based on the monitoring information, the policy server  may  decide to
migrate flows   from an overloaded   network function to   a different
instance, insert new network functions  in a supersession, e.g. --- introduce a DPI after  a light IDS flag packets
for deeper  inspection, or   remove a network
function  out of a  supersession path because it  is no longer needed.
To further extend system flexibility, network administrators can allow the policy server to offload decisions to the \system agents
running on the middleboxes.
A common decision that  can be easily offloaded to  the agents is  the
selection   of a     network   function instance   when  the   network
administrator configures multiple instances of the same type.

%The policy server also decides how a supersession is It determines how
%a supersession is initially divided into subsessions, i.e., it decides
%the middlebox chain a set of flows should traverse in order to enforce
%a network policy.

%- It outsources fine grain  decisions to the  middlebox agents. It can
%send to the middlebox agents a list of MB types to  from the MB chain,
%but the decision to pick an instance is outsourced to  the MBA. It can
%also provide autonomy to a MB to remove itself out the way of a flow.

In  \system,  a
policy  specifies a  predicate,  which defines a set of affected
packets, and a  sequence of network  function types.  This sequence defines the service  chain to be  traversed by the packets  satisfying the predicate, e.g. --- packets from subnet 10.0.1.0/24
should    be     forwarded       through      the     chain     \{Load
Balancer $\rightarrow$ IDS $\rightarrow$ Firewall $\rightarrow$ Proxy\}.    The general form
of a policy is:
\begin{center}
{\tt match(predicate) >> \{NFT$_1 \rightarrow\cdots\rightarrow$ NFT$_n$\}}
\end{center}
%\noindent where
%\begin{center}
%{\tt NFT$_i$ = \{Set of IP addresses\}}
%\end{center}
A predicate is specified with source and  destination IP addresses and
ports, and protocol types.   Complex predicates can be specified using
conjunction (\&),  disjunction  ($|$),  and negation (\~{})  operators
like in Pyretic~\cite{pyretic}. A packet that satisfies  the predicate in the
match  statement is forwarded  through  the chain of network  function
types specified  on the right-hand side of  the  {\tt >>} operator.  A
network   function type ({\tt NFT$_i$})   specifies  a set of  network
function instances  of the same  type. For example the network  administrator can
specify such a set of proxy servers, and
the policy  server will either choose an instance for each client or  delegate   this  decision to  the \system
agents.

%Predicates: The  predicates are specified  with respect to the point a
%packet enters  the  network,  i.e., the  point  of attachment    of an
%endpoint.


{\bf   \system Agents:} \system agents   are key enforcers of
network  policies.   They  must ensure    that  packets are  either correctly
forwarded to the next  middlebox on the  service  chain or dropped  if
they  do not   comply with  the  network policy.   Whenever  a  client
application starts a connection, the \system agent running on the same
machine intercepts the  first packet of  the connection and matches it
with  the   client's network  policies to  determine  the  sequence of
network function  types the packet must traverse.   If a match is
found, the agent  starts the session  setup protocol  in \S\ref{setup}
that   will create the     subsessions  between middleboxes  and   the
corresponding mappings as described below.   The 5-tuple of the client connection
is used to  identify the supersession  packets when they are processed
by  the  network  functions.   Once the  supersession  is   setup, all
subsequent  packets from the client's connection  are forwarded to the
first  middlebox  on  the service  chain,   and  from there  they  are
sequentially forwarded to the remaining middleboxes and to the server.

In each middlebox, a  \system agent must  create two mappings for each
session. One, which we call horizontal NAT,  translates the 5-tuple of
an incoming  packet to the corresponding 5-tuple  that is  used in the
next   subsession.  The   other one,  which   we  call vertical   NAT,
translates   the 5-tuple of each   incoming packet to the supersession
5-tuple before the  packet is delivered to the  network function or to
the applications on  the  endhosts.  Keeping the  supersession 5-tuple
invariant    on  the middleboxes  has  several advantages. First,  it
simplifies policy specification by abstracting the subsessions from the network administrator.  Second,  it simplifies network function state
migration, because network functions  always receive packets  with the
supersession  5-tuple   regardless of flow    migrations or  middlebox
insertions  or removals.  Third,  network functions that do not change
the packet 5-tuple do not need to be changed to work with \system.

\system agents are also heavily involved  in service chain maintenance
and  supersession  reconfigurations.   The  agents report   management
information, such as resource utilization  in the middleboxes, to  the
policy server, so  it  can  make  global management  decisions   about
service chains.  If allowed by network policy, reconfigurations can  also be triggered by a  network
function.   A  common case  of a
reconfiguration triggered by a network function is  when a cache proxy
detects that the content of a session is not cacheable and signals the
\system agent to  remove the  proxy  from the service  chain.  Once  a
reconfiguration is  initiated, either by the policy   server or by the
network function,  \system      agents  execute the  protocols      in
\S\ref{MigrateLogic}.

%{\bf End hosts:}

%\subsection{Division of Labor}

%Clean separation of control, data, and management 

%\subsection{Mechanisms}




% archiIllustrrate
\begin{comment}
\begin{figure}[hb]
\centering
% \includegraphics[scale=0.25]{figures/netfilter.pdf} 
\includegraphics[width=\linewidth]{figures/archiIllustrrate.pdf} 

\caption{\small Middlebox protocol architecture}\label{expTopo}
\end{figure}
\end{comment}

\section{Protocol}
\label{sec:protocol}
\begin{comment}
We explain the basic indirection protocol in subsection~\ref{basic}, a three-way handshake that exchanges state enabling future migration in subsection~\ref{twoway}, and a migration protocol in subsection~\ref{mobile}. The middlebox-aware session protocol can: (i) successfully establish a connection through the two-way handshake; (ii) support a flexible migration of either endhosts or middleboxes; and (iii) gracefully close the connection, or sub-sessions during the migration. 
\end{comment}

Considering middleboxes as explicit components of the end-to-end between two endpoints is the crux of our protocol. Only by doing so can we achieve the desired scalability and flexibility for both endpoints and middleboxes. We discuss session setup in \S\ref{setup} and flow migration control in \S\ref{MigrateLogic}.  

\subsection{Session Setup}\label{setup}

In \system protocol, each endpoint or middlebox sends packets whose destination is the next middlebox or endpoint in the session path. This obviates the need for special support in the switch or router to direct packets through the chosen chain of network functions (service chain), despite changes in network topology or host movement. 

The list of middleboxes, $L$, that a flow has to traverse is provided by the policy server and can be pulled from the server or pushed to the client. When the client initiates the connection, the control plane uses a three-way handshake to establish the supersession and its associated subsessions. More specifically, the client's control plane sends a SYN message that includes the supersession header and $L$ to the first middlebox. The middlebox strips itself from the head of $L$, gets the address of the next middlebox from $L$, and relays the rest of the message to the next middlebox. The SYN message is thus passed recursively through the elements of $L$ before reaching the server. Upon receiving the SYN, the server sends a SYNACK back to the client along $reverse(L)$. Upon receiving the SYNACK, the client immediately sends an ACK to the server via the same mechanism. Once these three control messages are exchanged, the supersession and subsessions are established, and data packets are explicitly addressed to the subsession IPs.

If we simply rewrite the source and destination IPs as described in our ``horizontal NAT'', we lose supersession information and introduce ambiguity. Consider the case where flow \textit{a} and \textit{b} have the same source port and destination IP and port, but different source IPs. If \textit{a} and \textit{b} share the same first hop middlebox, the two flows may become indistinguishable upon arrival at the first hop middlebox. To address this issue, we modify the port numbers to identify the flow, a standard technique in NAT~\cite{NAT}. We integrate this port allocation into the three-way handshake: when a middlebox receives a SYN, it assigns port mappings and initiates the next subsession with the rewritten port numbers. If we rewrite both source and destination ports, \system can support four billion unique flows per middlebox pair. See Figure~\ref{sessionsetup} for the complete session setup. \amy{ Can we insert some words over the port remapping in Figure 3, to indicate that the $\rightarrow$ represents port remapping?}

\begin{figure}[ht]
\centering
% \includegraphics[scale=0.25]{figures/netfilter.pdf} 
\includegraphics[width=\linewidth]{figures/threeway.pdf} 

\caption{\small  Session setup}\label{sessionsetup}
\end{figure}



\subsection{Migration and Mobility Control}\label{MigrateLogic}
%Dynamic network function policies are gaining ground today because of the flexibility they offer. 

Supporting dynamic middlebox modification for a flow improves the efficiency of the network and NF use, e.g. --- removing a cache proxy if the content is not cache-able, inserting an IDS upon detecting suspicious flows, or switching from a heavily loaded transcoder to a lightly loaded one. As a natural extension of general middlebox migration, we include endhost mobility in the design of \system; supporting mobility is also critical in cellular networks.

\subsubsection{``Make before break''} \label{migration1}
To find the right mechanism to support flow migration, we investigated existing mobility protocols~\cite{TCPMobile, I3Mobile, mip, serval, lisp, hip} under the session-location mobility framework~\cite{zave}. However, the key distinction between host mobility and NF flow migration is that a move in host mobility is unexpected, whereas flow migration among middleboxes is planned.

In fact, this type of reconfiguration is quite common in circuit design and addressed via the ``make before break'' philosophy. Namely, we stitch together subsessions on the new path before closing the subsessions on the old path. To achieve this, we treat the two neighbors of the moving middlebox as two \textit{signaling endpoints} during a flow migration. We stress that the resulting three nodes that are involved in the migration are \textit{consecutive}. Because we offload some policy decisions to the middleboxes, this property ensures that a middlebox cannot decide the fate of other on-path middleboxes that are not directly affected by the migration. 

For middlebox insertion, we first deterministically identify and notify one of the two signaling endpoints as an initiating point. Starting at the initiating point, we set the two signaling endpoints to a suspend state to prevent data transfer on the new path and complete a three-way handshake (UPDATE-SYN, UPDATE-SYNACK and UPDATE-ACK) on the new path, consisting of the two signaling endpoints and the new middlebox. Once the new path is established, we commence data transfer on the new path and remove the old path. We extend the same mechanism to handle middlebox removal and replacement. 

\subsubsection{Handling Concurrent Migration}
For correctness, we must properly deal with cases where two middleboxes initiate simultaneous move operations. In particular, if two neighbor middleboxes both initiate removal, we may lose the supersession connection. Strawman solutions include: (1) using a two phase commit to allow the one with the highest ID to move first; (2) a central controller that assigns token to the middleboxes. However, strong serialization via a two phase commit is not necessary, and a central controller suffers from scalability, one of our initial design goals.

To address concurrent migration, we rely on the following properties: (i) migration is per flow, and (ii) at most three nodes on the current service chain are involved in each flow migration. The two assumptions help us design an efficient algorithm in which migration can happen simultaneously for different flows, and concurrent migrations can happen at different points of the service chain if they involve disjoint sets of nodes; each node has a \texttt{pending} counter to ensure that it participates in at most one concurrent update.


The proposed algorithm only allows one concurrent operation for all the nodes involved. When a migration is initiated at a certain node, the node postpones the update if its \texttt{pending} counter is not zero. Otherwise it increments the \texttt{pending} counter and sends the request to the node immediately to its left if the migration is removal or replacement, or connects to the new middlebox if it is an insertion. Middleboxes for a flow have a strict ordering, which is simply the order in which the middleboxes are traversed by the path from the client to the server. We define two comparators on this ordering, which we term ``left'' and ``right''. $M_{1}$ is left of $M_{2}$ iff the path from the client to $M_{2}$ passes through $M_{1}$; $M_{1}$ is right of $M_{2}$ iff the path from the client to $M_{1}$ passes through $M_{2}$. If the left node responds with a reject, the node backs off, otherwise it receives an approval and proceeds with the update. A node always accepts UPDATE-SYN requests to establish a new path. Algorithm~\ref{concurrency} describes the algorithm details and Figure~\ref{concurfigure} depicts the steps for migrations.


\begin{figure}[ht]
\centering
\includegraphics[width=\linewidth]{figures/concurrentupdate.pdf} 
\caption{\small Flow migration during insert, remove and replace}\label{concurfigure}
\end{figure}


\begin{algorithm} [htb]
%\small
\scriptsize

\SetAlgoLined

\SetKwFunction{update}{Trigger\_Migration}\SetKwFunction{IPC}{Msg\_Handler }\SetKwFunction{queue}{Queue Agent}
\SetKwProg{func}{Function}{}{}

\func{\update{} } {
\If{recv(migrate)}
{
  \If{pending == 0} {
  pending++\;
  \If{migration == insert }{
    sendto(New right, UPDATE-SYN)\;
   
  } \Else{
   sendto(Left, request)\;
  }
  }
  \Else{
  Exponentially backoff and retry\;
  }
}


}
\func{ \IPC{} }{
\If{recv(request)}{
  \If{pending$>0$ }{
  sendback(reject)\;
  }
  \Else{
  pending++ \;
  sendto(New right, UPDATE-SYN)\;
  sendback(approve)\;
  }
}

\If{recv(reject) }{
 Exponentially backoff and retry sendto(Left, request)\;
}
\If{recv(approve) }{
//do nothing, to avoid request re-transmission
}

\If{recv(UPDATE-SYN)}
{
  pending++\;
  \If{(migration==insert or replace) and (current == not signaling point) }{
    forward(UPDATE-SYN)\;
  }
  \Else{
  sendback(UPDATE-SYNACK)\;
  } 
  
}
\If{recv(close)}
{
  //clean old flow state\;
  pending$--$\;
}
\If{recv(UPDATE-SYNACK)}{
  pending$--$\;
  \If{(migration==insert or replace) and (current == not signaling point) }{
    forward(UPDATE-SYNACK)\;
  }
  \Else{
  sendto(OldMBox, close)\;
  sendback(UPDATE-ACK)\;
    //clean old flow state\
  } 
  
}
\If{recv(UPDATE-ACK)}{
  pending$--$\;
   \If{(migration==insert or replace) and (current == not signaling point) }{
    forward(UPDATE-ACK)\;
  } \Else{
    //clean old flow state\
  }
 
}

}
\caption{ Concurrent Flow Migration} \label{concurrency}

\end{algorithm} 


\subsubsection{``Break before make''}
\amy{Do you mean ``supersession'' in the following bolded words?}
When a client moves, it may drop the old \textbf{subsession} before establishing a new \textbf{subsession}. Consider when a UE moves across a cell boundary, upon which the UE may suffer from transient connection loss, since it is out of the old cell's range. 

After losing the old \textbf{subsession}, the client needs to rebind to the first hop middlebox. If we use the client's physical IP as part of the supersession identification, the first hop middlebox will fail to identify the supersession if the client changes its IP during mobility. To solve this problem, we can either put the old connection's information in the rebinding message sent by the client or, in a single domain case, administrators can assign a non-routeable IP to each device as a unique ID and use this ID to help identify the supersession. 
 


\section{Data Plane Properties}
In this section, we discuss three data plane properties and the mechanisms that support them atop the control logic. \amy{the data plane is technically below the control logic. Maybe find another word for ``atop''}

\subsection{Loss-Free Update}

%In the migration control logic, we have the concept of ``path''; let us now only focus on data plane and see how path is reflected there.

At the data plane, there is a translation table stored in each middlebox. The \system agent accepts flows for which it has established state, rewrites the header, and forwards the flow based on the supersession-subsession mapping stored in the translation table. Moving a flow from one path to another is equivalent to updating the translation tables at the data plane to accept a flow from the new subsession(s) and reject the same flow from the old subsessions. Since there are multiple middleboxes involved in the update procedure, if the update happens in the wrong order, the translation layer may drop packets, e.g. --- if the old subsession is removed before the new subsession is fully established.


Finding the right sequence of updates for a general network is proven to be NP-complete~\cite{SWAN, zUpdate}, but for a special topology, linear in our setting (we only abstract the topology between middleboxes), we can apply the concept from network consistent update~\cite{consistentupdate, ratul}. We first ensure that all new translation rules are pushed before the egress applies the new rule, and the old rules are not removed until all the new rules are installed via a complete handshake. In particular, when a migration is initialized from the middlebox (a signaling point), the middlebox notifies its neighbor through the control plane, which inserts a new rule for incoming traffic. Then this neighbor notifies the other side of the connection with an UPDATE-SYN control message. Every hop that receives UPDATE-SYN updates its own translation table. Hence once the other signaling endpoint receives the notification, a new rule for the flow has been installed at every hop for one direction of traffic; it is thus safe to apply the new rule for the egress. The opposite direction is set up in the same way with UPDATE-SYNACK messages. Once the new bidirectional path is built, we tear down the old path by removing the old rules. This loss-free update mechanism mirrors the control plane ``make before break'' philosophy and is in fact facilitated by the control plane design. 

 \begin{figure}[ht]
\centering
\includegraphics[width=\linewidth]{figures/order_preserving.pdf} 

\caption{Flow Migration in Packet Order Preserving ((a) and (b) can happen in parallel.)} \label{orderpreserving} 
\end{figure}
 
\subsection{Packet Order Preserving} \label{FIFO}
The previously described loss-free update is insufficient if the NF instance (state) also needs to migrate. More specifically, the NF state cannot be migrated since it is being continuously updated as packets are coming in via the old path. To lock and migrate the middlebox state, \system must stop sending traffic during the new path setup. Since changing the protocols' (e.g., TCP) flow control is undesirable, we choose to buffer the traffic and do not release it until the network function state has been replicated at the new path middlebox. Since NF state replication and migration is a well solved problem~\cite{OpenNF, splitmerge, HAMbox}, we do not address this problem here. 

A complete flow and state migration takes 10 steps as depicted in Figure~\ref{orderpreserving}: 
1) lock outgoing traffic; 
2(a) send UPDATE-SYN via new path; 
2(b) send UPDATE-FIN via old path;
3) lock the reverse direction traffic;
4(a) send UPDATE-SYNACK via new path;
4(b) send UPDATE-FINACK via old path;
5) lock middlebox states;
6(a) send UPDATE-FINACK via old path;
6(b) migrate states;
7) unlock middlebox states;
8) send UPDATE-SYNACK via new path;
9(a) send ACK packets;
9(b) release buffered traffic;
10) release buffered traffic for the other direction.
\newline \amy{is this newline intentional}


\begin{figure}[ht]
\centering
\includegraphics[width=\linewidth]{figures/opennfbroke.pdf} 
\caption{\small Order-preserving problem in OpenNF without FIFO assumption, the signal packet \texttt{s} and the data packets 1 and 2 maybe reordered and thus create reordering. Note the switch is one big switch abstraction.}\label{opennfbroke}
\end{figure}


This update mechanism also satisfies the same \textbf{packet order-preserving} property as defined in OpenNF: assuming the path with the NF instance is \textbf{FIFO}, i.e., the control message sent after the last data packet is always received after the last data packet, \textit{all packets should be processed in the order they were forwarded to the NF instance by the switch (network)}~\cite{OpenNF}. We will now show how we achieve the same property without the FIFO assumption.
 
\subsection{Substream Order Preserving}  

During migration it is useful to create two separate substreams of a byte stream (e.g., TCP), one each for the old and new path. For example, when migrating a flow from one IDS to another, we may want to ensure that all the SYNs and corresponding ACKs go through the same IDS in order to avoid an alert like ``ACK before SYN''. When the flow passes through a deep packet inspection (DPI) NF, both string matching and reg-exp matching build a Deterministic Finite Automaton (DFA), which is traversed from the root based on the byte stream~\cite{aho, yaron}. If we want to use different DPI instances for a byte stream, but the first instance has not seen a complete substream, the first DPI can only check the largest continuous byte stream before it has to buffer the remaining bytes and send them to the second DPI. 


In order to divide substreams cleanly, the left neighbor of the moving node leverages TCP sequence numbers. Suppose a migration is initiated at time $t$. The left neighbor uses the maximum seen TCP sequence number at time $t$ as a \textit{checkpoint}: it buffers packets with a higher sequence number than the checkpoint and forwards packets with a lower sequence number. \system does not lock and migrate the old NF instance state until it sees a TCP ACK with sequence number \textit{checkpoint}. The ACK guarantees the delivery of all the packets in the old substream. Note that although the buffering may create a reordering at the left neighbor middlebox, this protocol results in the correct order for the two substreams from the perspective of endpoints. See Algorithm~\ref{strictorderpres} for details. 


%%%%%%%%%%%%%%%%%%%%%%%%Algorithm %%%%%%%%%%%%%%%%%%%%%%%%%%%%%%%%%%%%%%%%%%%%
%%%%%%%%%%%%%%%%%%%%%%%%%%%%%%%%% %%%%%%%%%%%%%%%%%%%%%%%%%%%%%%%%%%%%%%%%%%%%
%%%%%%%%%%%%%%%%%%%%%%%%%%%%%%%%%  %%%%%%%%%%%%%%%%%%%%%%%%%%%%%%%%%%%%%%%%%%%%
%%%%%%%%%%%%%%%%%%%%%%%%%%%%%%%%%  %%%%%%%%%%%%%%%%%%%%%%%%%%%%%%%%%%%%%%%%%%%%
%%%%%%%%%%%%%%%%%%%%%%%%%%%%%%%%%  %%%%%%%%%%%%%%%%%%%%%%%%%%%%%%%%%%%%%%%%%%%%

\begin{algorithm} [htbp]
\footnotesize
\SetAlgoLined
\SetKwFunction{syn}{recv\_SYN}\SetKwFunction{ack}{recv\_ACK}\SetKwFunction{queue}{release\_queue}
\SetKwProg{mypacket}{Event\_Handler}{}{}
\SetKwProg{func}{Program}{}{}

\mypacket{\syn{TCP\_packet p} } {
checkpoint = hash\_lookup(p)\;
\If{p.seq $>$ checkpoint} {
Buffer (p)\;
} 
\Else{Forward(p)\;}
} 


\mypacket{\ack{TCP\_packet p}}{
\If{p.ack $>$checkpoint}{
Migrate NF state\;
//wait until migration finishes\;
sendto(left neighbor, release\_buffer)\;
}
}


\mypacket{\queue{}}{
\While{!buffer.empty()}{
Forward(buffer.dequeue())\;
}
Reset(hash\_table)\;
}

\caption{Substream Order Preserving} \label{strictorderpres}
\end{algorithm} 


%%%%%%%%%%%%%%%%%%%%%%%%Algorithm %%%%%%%%%%%%%%%%%%%%%%%%%%%%%%%%%%%%%%%%%%%%
%%%%%%%%%%%%%%%%%%%%%%%%%%%%%%%%% %%%%%%%%%%%%%%%%%%%%%%%%%%%%%%%%%%%%%%%%%%%%
%%%%%%%%%%%%%%%%%%%%%%%%%%%%%%%%%  %%%%%%%%%%%%%%%%%%%%%%%%%%%%%%%%%%%%%%%%%%%%
%%%%%%%%%%%%%%%%%%%%%%%%%%%%%%%%%  %%%%%%%%%%%%%%%%%%%%%%%%%%%%%%%%%%%%%%%%%%%%
%%%%%%%%%%%%%%%%%%%%%%%%%%%%%%%%%  %%%%%%%%%%%%%%%%%%%%%%%%%%%%%%%%%%%%%%%%%%%%


A combination of order preserving and substream separation provides us with a stronger \textbf{substream order preserving} property during an update: \textit{substreams should be processed by different NF instances in the order they were sent from the sender}. OpenNF cannot guarantee order preserving with respect to a byte stream when network links are not FIFO~\footnote{OpenNF tech report relies on this assumption in its proof.}. Furthermore, because OpenNF is by design router-based, it simply \textit{cannot} achieve substream order preserving; see Figure~\ref{opennfbroke}. Our protocol is able to provide this property precisely because the system architecture is designed to be aware of and rely on transport protocols.


Note that we may need to split the packets in the case of SYN-ACK piggyback and packet coalescing. In the first case, both directions asking for substream separation can result in deadlock since \system may choose the new path for a substream with higher sequence number in one direction and the old path for ACK as it is ACK-ing a substream with lower sequence number in the reverse direction. In the second case, the TCP client may coalesce packets, causing the payload  to cross the \textit{checkpoint} boundary in the byte stream if retransmission occurs. 








\section {Implementation}

\subsection{Prototype}

There are    three  options   for  an in-kernel
implementation of the data plane: (i) OS  native  network stack; (ii) NIC
driver (e.g.,  DPDK~\cite{dpdk});   and  (iii)   customized  in-kernel
software switch (e.g.,  openvswitch~\cite{ovs}).  We chose option (i) because (a) \system has  to choose the  right interface before sending to
the driver   when  having  multiple  interfaces;  and  (b)  a customized
in-kernel software switch has high overhead.  We implement  a kernel module data  plane  and user  space control/management
plane via TCP/UDP sockets, with a total of about 4000 lines of code in
C  and   C++.   We install kernel modules to
register callback  functions with  Linux  \texttt{netfilter}  for data
plane operations. The   control/management  plane  has    extensible,
complex control logic, and the data plane does specific, simpler actions
such as header rewriting, queuing packets to user  space via \texttt{netfilter\_queue} and  updating the
kernel hash table, which holds the flow $\rightarrow$ port mapping.  The user and the kernel  agent  communicate via 
\texttt{netlink}, a native   Linux  Inter Process Communication  (IPC)
function.  Currently the policy server  is   a simple TCP server  that
proactively pushes policies to the agents.

\begin{figure}[ht]
\centering
% \includegraphics[scale=0.25]{figures/netfilter.pdf} 
\includegraphics[width=\linewidth]{figures/archiIllustrrate.pdf} 

\caption{\small Layout of the implementation blocks}\label{netf}
\end{figure}





\subsection{Network Function Support}\label{sec:NFsupport}

We categorize NFs into two types --- active and passive functions --- based on whether the NF acts on the original packet.  

\begin{table}[ht]\label{middleboxextension} 
\centering
 
\small
\begin{tabular} {|l |c |c |c|}
\hline

      Name          	  &         Type           & Key        	 &      Binding   \\
                      	  &                        &  Functions          &       Library  \\ \hline
PRADS~\cite{prads} (P) 	  &      Monitoring        &    got\_packet()      & libpcap   \\ \hline
Bro~\cite{bro} (P)      	  &      IDS               &   DumpPacket()      & libpcap   \\ \hline
Snort~\cite{snort} (P)  	  &        IDS         &    PQ\_Show()          & libpcap \\ \hline 
Balance~\cite{balance} (A)	  &      Load Balancer     &    recv(), writen()       &user socket\\ \hline
Squid~\cite{squid} (A) 	  &        Proxy           &  getsockopt()        & user socket  \\ \hline
Traffic- 		  &    WAN-                &     net\_receive       &Linux  \\ 
Squeezer~\cite{tsqueezer} (A)&Optimizer &\_skb()   & skbuff \\ \hline


\end{tabular}
\caption{\small Commonly-used Network Functions; (A) means active and (P) means passive NF. }\label{nfhook}
\end{table}


 Many passive NFs make decisions based on a clone of the original packet. Based on our survey in Table~\ref{nfhook}, we found that most passive NFs use libpcap to capture cloned packets from a raw socket. As described in \S\ref{sec:arch}, we must restore the supersession header of the copies before delivering them to the NF. 
To implement this so-called vertical NAT, we modified the \texttt{pcap\_handle\_packet\_mmap()} function in \texttt{pcap-linux.c} in libpcap~\cite{tcpdump} to restore the supersession. 

For active NFs, there are two cases. If the NFs only act on the payload or MAC layer, e.g.--- TrafficSqueezer, supersession restoring also works. 
However, if the NF acts on the 5-tuple, we have to extend NF functions to notify the \system agent of its header mapping. 
For example, a transparent cache proxy~\cite{squid} gets the packets from its listening port and sets up a new TCP socket to the final destination. \system will break the supersession into two if it is unaware of the mapping between two sessions. On the other hand, if the NF informs the \system agent of the mapping between its listening and sending TCP sockets, the \system agent can stitch the two subsessions into the same supersession. 
 



% \subsection{Middlebox Support}




% 
%Three way handshake
%read write lock
%kmalloc atomic for faster access


\section {Evaluation}

In this section we  show that  the  separation of the data  plane from
control and management operations in  the \system architecture results
in a  system where network functions  can be deployed with  very small
overhead.   Compared with a    conventional deployment that  relies on
routing for  service chaining, \system adds a  network round trip time
for session setup, which  can be  easily  and safely eliminated  in an
enterprise network if the  session setup information is piggybacked in
TCP  syn    packets,      we  describe      this optimization       in
\S\ref{sec:3way-handshake}. Moreover, we show that \system can sustain
throughput  at line  speed    in a   40   Gbps testbed  and    present
microbenchmark results for  the overheads introduced for table lookup,
rule insertion, and header rewrite.   We also show measurement results
for latencies introduced during  session  setup when multiple  network
functions are on the  session path and  latencies for flow  migrations
under different order preserving semantics.


%In our evaluation, we demonstrate that:
%\kelvin{Need to add more eval!}
%\begin{itemize}
%\item The system can sustain very high throughput
%\item The system has extremely low latency for flow migrations
%\end{itemize}

We used two  dedicated testbeds for  the performance  evaluation, with
the following configuration:  (i) one L2 switch  with 16  1Gbps ports,
and five  {\em mid-range} workstations,  each with one Intel Quad-Core
Xeon 3.7GHz processor,  32GB of memory, and two  1Gbps NICs; and  (ii)
one L2 switch with four 40Gbps ports, and four {\em high-end} servers,
each with two Intel 8-Core Xeon 2.2GHz processors, 64GB of memory, and
one  40Gbps NIC.   We  conducted throughput  stress tests in  the four
high-end servers, and unless specified, all the other experiments were
performed in setting (i).

\subsection{Throughput }

We first show that  \system can  sustain throughput  at line speed  in
40Gbps links using a vanilla Linux kernel and  1500-byte packets.  The
throughput  numbers  were  obtained  from   {\tt   iperf}~\cite{iperf}
executions  on  three  {\em high-end}  machines   connected  on a line
topology client--middlebox--server.  Since  the high-end machines have
only one NIC  each, we configured virtual interfaces  on  the NICs and
connected them to different IP subnets.  As a baseline for comparison,
we  enabled  IP routing  on the  middlebox  on  the  same topology and
removed the \system kernel modules on the three machines. We also show
the throughput results for  three other  configurations: first, a  VPN
tunnel  between  the middlebox  and the  server; second,  the same VPN
tunnel with compression enabled (VPN+OP); and third, a software switch
(OVS) performing network address  and port translations.   VPN tunnels
have been proposed  as a solution  for outsourcing virtualized network
functions to the cloud~\cite{Aplomb}.  The OVS results are included to
show the throughput gains of  having a dedicated kernel implementation
for \system.


\begin{figure*}[ht]

\centering
% \includegraphics[scale=0.25]{figures/netfilter.pdf} 
\includegraphics[width=\linewidth]{figures/throughput.pdf} 

\caption{ Throughput  of  different approaches.}\label{throughput}
\end{figure*}

Figure~\ref{throughput} shows the throughput measurements for the five
different scenarios:  \system, baseline,  VPN, VPN+OP,  and  OVS.  The
results show the  \system kernel-module data plane  implementation can
sustain $14.2$ Gbps on  a single core,  and it scales linearly  to the
number of  cores, reaching  $37.1$ Gbps at  its peak.   The difference
betweek the peak throughput and 40Gbps is because {\tt iperf} measures
the throughput  at  the  application    layer, i.e., packet    headers
(TCP+IP+Ethernet) are not  included in the computation.   The baseline
case yields  a  throughput of  $14.6$  Gbps on  a  single core, so the
kernel module overhead  is under  $3\%$ in  the  worst case.   As  the
number of flows    increases,   more cores  are involved   in   packet
processing and the gap between \system  and the baseline shrinks until
it becomes negligible.  The reason  is that interrupts per core happen
less frequently and the link  bandwidth becomes the bottleneck instead
of the CPU. We envision if  we increase the NIC  bandwidth to 100 Gbps
or higher,  the gap will stay the   same and the throughput  will grow
linearly before    the link  bandwidth   or the   PCI bus  become  the
bottleneck.

We initially observed inconsistent results  for four flows because the
default  hash  function (Toeplitz  hash)   of the NIC  driver  was not
spreading  well the flows  into different  cores.   We had to apply  a
minor tweak  on the NIC driver to  get a  better distribution of flows
into cores.  When  the number  of  flows is small, the  driver  manual
recommends the  change of the Toeplitz hash  function to  a simple XOR
function to avoid   flow colliding on the   same core~\cite{mellanox}.
After we changed the   hash function, we observed consistent  results.
The same configuration was used for the five scenarios.

The throughput results for the other three scenarios are also depcited
in Figure~\ref{throughput}. We compare our approach with off-the-shelf
VPN-encapsulation mechanism,  which is used in APLOMB~\cite{Aplomb} to
achieve its redirection.  Not surprisingly,   VPN uses encryption  and
thus offers much lower  throughput.  OpenVPN~\cite{openvpn}, which  is
used in APLOMB, achieves  less than $400$ Mbps on  a single core, even
with the compression  optimization.  Current OpenVPN does  not support
multi-threading and is not  able  to take  advantate of  the  multiple
cores  on  the machines.  Even  if  multi-threading is incorporated in
OpenVPN in the future and assuming it will scale linearly to 16 cores,
it will be able to offer only 17\% of the maximum throughput.

OVS results.

%It is well   documented that when the  number  of flows is  small, the
%default  hash function of   the NIC driver   (Toeplitz hash)  does not
%perform well and  different flows  are mapped to   the same core.   We
%changed the   hash  function to a    simple XOR function,  which  is a
%recommendation of the driver manufacturer~\cite{mellanox}.

\begin{figure}[ht]
\centering
% \includegraphics[scale=0.25]{figures/netfilter.pdf} 
\includegraphics[width=\linewidth]{figures/CPU.pdf} 
% \includegraphics[width=\linewidth]{figures/cdf.pdf} 
\caption{\small Average CPU utilization in stress test. Note we only include the cores that are interrupted by the NIC.}\label{cpu-utilization}
\end{figure}

Figure~\ref{cpu-utilization}   shows  the    CPU  utilization  of  the
middlebox in the same experiments.  Note that since the middlebox acts
as a software  router without the  kernel module,  it also consumes  a
certain  amount  of CPU  cycles.   The highest difference between  the
baseline and the system  lies  in the setting  of   4 flows,  the  CPU
utilization  for \system  is 10\%   higher than that  of the  baseline
(27.6\% versus 17.2\%), but under most circumstances the gap is within
2\%. One thing worth noting is that  as the number of flows increases,
the  CPU utilization per core  plummets, our explanation is that since
the testbed machine has 16 cores and 32  hyper-threads, it spreads out
quite evenly, and the CPU utilization scales linearly at high load but
it is not the case at low load  (e.g., 1M packets/s causes much higher
CPU utilization than 500K packets/s).

The  experimental   results    are  promising   and   prove  that  the
kernel-module data plane implementation can offer very high throughput
at the cost of modest increase of CPU utilization.


\subsubsection{Microbenchmarks}

The kernel-module implementation of the data plane includes three main
functions:  (i) table lookup for   vertical and horizontal NATs;  (ii)
header rewriting and re-checksum; and (iii) rule installation.

We also
have  a  few  optimizations in  the  data-plane  implementation (e.g.,
partial checksum to reduce CPU cycle, a small hash  table to fit in L3
cache)    result in an extremely  efficient   system with   a very low
overhead.

We conducted a microbenchmark to see the delay different functions add
to the  system. We insert 100K  rules  in the  kernel  hash table, and
conducted 100K  lookup. We also  microbenchmarked the header rewriting
and partial checksum of  the data packets.   We had the  two following
methods to  eliminate Linux timer's  overhead: (i) batch  and time the
insertion  and lookup for every one  hundred rules; (ii)  get the pure
timer lookup time and then subtract that base number. The result shows
that lookup, insertion and header rewriting takes on average 131, 251,
and 214 ns, this  gives us $\approx  $  2.8M packets per seconds  with
lookup and header rewriting, this is  equivalent to 33.6 Gbps per core
with MTU of 1500 Bytes.

\begin{figure}[ht]
\centering
% \includegraphics[scale=0.25]{figures/netfilter.pdf} 
% \includegraphics[width=0.53\linewidth]{figures/CPU.pdf} 
\includegraphics[width=\linewidth]{figures/cdf.pdf} 
\caption{\small Time CDF for different functions}\label{microbenchmark}
\end{figure}
% 



%We also  changed the NIC's ring buffer
%hash   function to an   XOR   hash function   to avoid  core interrupt
%collision, since   the default hash   function  is designed  for large
%number of flows, we see noticeable collision,  i.e., hash two flows to
%the same ring buffer when the number is small.
%The gap is closing as  we increase  the number of
%flows,  sice it  incurs    less frequent  interrupts   per core  (the
%frequency of  per-core interrupt for 4 flows  with 9 Gbps  per flow is
%66\% of a single flow with 14.5 Gbps), and thus the stress per core is
%lighter. 


%One interesting  observation is that when there   are three flows, the
%system offers a higher throughput than the baseline. The reason behind
%it  is that the ``extra  work'' tends to keep  the cores busy and thus
%gets more  CPU  cycles for  processing.  The competing flows from  TCP
%congestion control may also affect the throughput.

\subsection{Latency}

We evaluate our system's latency in the  following order: (i) how much
overhead it has  for a super-session setup? (ii)  how much overhead it
has during an update while preserving  different properties. We choose
setting (i) testbed  since it is closer to  a real  enterprise network
w.r.t.  round trip time and switch  buffer size (queuing delay). Since
\system  is  per-flow based  operation, the number  of flows  does not
affect per-flow latency.


\subsubsection{Three-Way handshake}
\label{sec:3way-handshake}
% \kelvin{will add more stuff, like three hops, 4 hops and 5 hops, but do get the idea of how to do it now, three hops values are 426.5 28.7515216989, 464.5 22.8527897641, 688.1 24.6473122267}

We first measure the latency for  session establishment, in a topology
of three  hops (client -  middlebox -  server). We have  two different
implementation, (i)  out of  band UDP  ahead   to TCP handshake,  (ii)
piggyback  TCP handshake   at  the   payload. Not surprisingly,    TCP
piggyback can   greatly reduce the extra    latency due to propagation
delay, with a modest increase less than $40 \mu $s. On the other hand,
out-of-band UDP incurs $688\mu$s delay ahead of TCP handshake.

\begin{figure}[ht]
\centering
\includegraphics[width=\linewidth]{figures/latency_threeway.pdf} 
\caption{\small Latency for session establishment}\label{threeway}
\end{figure}


\subsubsection{Flow Migration}

\system supports flow  migrations with various properties.  We measure
the time  interval for   such  an update   with a  loss-free,   packet
order-preserving,  and  substream order  preserving.  We  also created
artificial   packet  reordering in  the  case   of  byte stream  order
preserving during an update.


We run our experiments in a topology with four machines (A,B,C,D) with
two paths  (A - B -  D) and (A  - C - D)  that  each consists of three
hops. Client and server are at A and D. The goal of the flow migration
is   to replace  middlebox B   with  D. We   measure  the latency  for
migrations with (a) loss-free (LF), (b) order-preserving (OP), and (c)
substream separation   and order  preserving  (SS+OP)  properties. The
latency are  evaluated under a full  load on a  network with all 1Gbps
links. All tests are run in TCP Cubic.

\begin{figure}[ht]
\centering
\includegraphics[width=\linewidth]{figures/latency_four_types.pdf} 
\caption{\small Latency of flow migration with loss-free (LF), packet order-preserving (OP), substream-separation and order-preserving (SS+OP), and substream-separation and order-preserving (SS+OP) with an artificial reordering. }\label{fig_update_latency}
\end{figure}


We summarize  the  latency result  on Figure~\ref{fig_update_latency}.
During a loss-free migration, the old path  is saturated; however, the
new path  is idle during  a subsession  setup and thus  has the lowest
latency. During a  order  preserving migration,  \system  has to  send
subsession setup on the new path and a marking  packet right after the
end of data packets on the old path; the higher  latency is mostly due
to the queuing delay on the old path.  During substream separation and
order preserving migration, besides the queuing delay, the ACK message
may also be delayed from the receiving side, e.g.,  the extra time for
TCP to process  the byte stream  and generate an  ACK. Since we rarely
see packet reordering in the testbed during an update, we artificially
created a  packet  reordering  via dropping a    packet  with a  lower
sequence number in the kernel, which triggers  a retransmission of the
packet  with a lower  sequence number after  the  packet with a higher
sequence number.

The results demonstrate that the  system can deal with flow migrations
in  the order  milliseconds.   This can greatly  facilitate   the flow
migration during an NF insertion, removal and hot-standby replacement.
NF      state  migration      can     happen  within     hundreds   of
milliseconds~\cite{OpenNF, splitmerge},  here results show that a flow
migration should no  longer be considered a bottleneck  in the case of
NF state  and corresponding flow  migration. A more detailed breakdown
is in Table~\ref{latencycomp}.

\begin{table}[ht]
\centering
\small
\begin{tabular} {|l |c | c| c|}

\hline
 &LF& OP &SS+OP \\ \hline 
% &~\cite{SIMPLE,FLOWTAGS, OpenNF}& ~\cite{CoMB, splitmerge}& ~\cite{ Aplomb,DOA,I3} & ~\cite{OVS, Netfilter} & \\ \hline

Split/Merge
\footnote{\scriptsize at packet rate 37.5 kpps, time are consumed across flow migration and state migration} 
		&~500ms& {\small no support}& {\small no support} \\ \hline
Dionysus
\footnote{\scriptsize Large-scale networks}
	      &600-1200ms &{\small no support}&{\small no support}   \\ \hline
OpenNF
\footnote{\scriptsize flow update time at packet rate 2.5-10 kpps} 
	       & 84ms& 96ms&{\small no support}\\ \hline
\system
\footnote{\scriptsize at packet rate 80 kpps (1 Gbps)} 
	    & 0.5ms & 2.7ms&3.6ms \\ \hline

\end{tabular}
\caption{Comparison with other dynamic middlebox and flow migration systems}\label{latencycomp} 
\end{table}



\section{Discussion}
\label{sec:discussion}

incremental deployment

security

fault tolerance





%%%%%%%%%%%%%%%%%%%%%%%%%%%%%%
%%%%%Related work
%%%%%%%%%%%%%%%%%%%%%%%%%%%%%%
%%%%Comparison table %%%%%%%%%

\begin{table*}[ht]\label{compare} 
\centering
\small 

\begin{tabular} {l |c | c| c| c| c }


                            			     &Scalable Fine-              &   Low Performance    &  NF's  Flow      &  Host        &  Incremental \\
						     &Grained NF Policy    	  &   Overhead           & Migration       &  Mobility   & Deployment \\\hline
						     
NF Policy                                           & TCAM size,      	          &                      & not discussed      &   not discussed  & inter-domain  \\
Enforcing~\cite{SIMPLE,FLOWTAGS}                   & TCAM update speed          &                      &                   &               &  middlebox \\ \hline


NF Dynamic                                          &TCAM size,     	          &  Handle packets      &                &  not discussed    & inter-domain\\ 
Control~\cite{OpenNF,splitmerge}                       &  TCAM update speed         & at the controller     &                &                & middlebox  \\   \hline


Naming service                          &			  	 & VPN, or per packet encap-                   & not discussed       & not discussed &  [some] new naming system  \\ 
based~\cite{DOA, Aplomb}                         &                              & sulate and identify          &                    &                  & new socket abstraction     \\\hline 



\end{tabular}
\caption{\small Reasons that different NF policy steering and mobility solutions failed to fulfill the properties}
\end{table*}





\section{Related work}

\textbf{Middlebox in SDN:}
SIMPLE~\cite{SIMPLE} and FlowTags~\cite{FLOWTAGS} take advantage of the switches with a fine-grained rule support in software-defined network (SDN), and support network function policy chaining in traditional and NFV setting respectively. Both approaches are constrained by the TCAM size, a hardware limitation in terms of fine-grained policy, and neither not support NF migration or host mobility. 

OpenNF~\cite{OpenNF} and Split- Merge~\cite{splitmerge} leverage the SDN controller to manage NF's state migration and NF's flow migration. However, since the central controller is involved in both control plane (update the network rules) and data plane (buffer the packets on the fly during migration), they suffer from high latency and low scalability. They also suffer from hardware limitation for fine-grained policies.  

\textbf{Middlebox using naming service:}
DoA~\cite{DOA} uses a delegation-oriented global naming space architecture, that explicitly specifies intermediary middleboxes on the path. Two key distinction between DoA and \system is that (i) \system does not require any new naming service and (ii) DoA does not support dynamic policy service. APLOMB~\cite{Aplomb} outsources the network functions to the cloud with a naming service indirection. It uses VPN to tunnel the traffic to the cloud and use DNS-based indirection to decide which cloud to enforce the middlebox policy. It assumes the cloud can provide elastic NF service but it cannot explicitly handle dynamic policy chain. 


\textbf{Middlebox consolidation:}
CoMB~\cite{COMB} and click~\cite{ClickOS, click} both consolidate network functions as an application or a VM image, and one host machine can run multiple instances of different network functions. Both approaches are mainly focused a feasibility and scalability of network functions on a single generalized servers. Both solutions consider neither scale-out across servers nor NF and flow migration.

\textbf{Mobility Protocols:}
Mobility protocols use (i) a fixed indirection point (e.g., Mobile IP~\cite{mip}), (ii) redirecting through DNS (e.g., TCP Migrate~\cite{TCPMobile}), (iii) indirection infrastructure (e.g., ROAM~\cite{I3Mobile}) or (iv) indirection at the link layer (e.g., cellular mobility). None of them consider the existence of middleboxes. Coexistence of network functions and new protocols is especially important for deployment, as a study in multipath TCP shows~\cite{MPTCP}. \system shares some similarity with many mobility protocols and has support for network functions.   

\small
\bibliographystyle{acm}

\bibliography{ref}

\end{document}
